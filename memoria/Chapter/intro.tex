\chapter{Introducción}\label{Int}
\fancyhf{}% Clear header/footer
\fancyhead[L]{INTRODUCCIÓN}
\fancyfoot[L]{\thepage} 
\section{Introducción}
En el presente documento se recoge una definición detallada del proyecto, así como los antecedentes y su justificación, el desarrollo, la planificación llevada a cabo para su realización, junto con el presupuesto empleado y finalmente las conclusiones y los
resultados obtenidos tras el proyecto. El documento se desglosa en los siguientes capítulos:
\begin{itemize}
    \item \textbf{Introducción: }
    Presentación de los contenidos del proyecto y de la estructura de
apartados.
    \item \textbf{Antecedentes y Justificación: }
    Una descripción del entorno del proyecto y del estado del arte actual en cuanto a aspectos técnicos, económicos y sociales, junto con una justificación técnica de la ejecución y viabilidad del proyecto.
    \item \textbf{Objetivos y Alcance: }
    Establecimiento del objetivo principal del proyecto, así como su alcance. Se nombran los aspectos funcionales que incluye y los que excluye. Estos objetivos han estado muy presentes a lo largo de todo el proyecto, desde las primeras fases de diseño hasta las últimas etapas de desarrollo.
    \item\textbf{Requisitos: }
    Identificación de los requisitos del proyecto.
    \item \textbf{Desarrollo: }
    En este apartado se describe el desarrollo, para implementar eficazmente el software, a presentar como proyecto final de grado.
    \item \textbf{ Planificación: }
    Descripción detallada del equipo de trabajo, enumeración y
especificación de las actividades realizadas junto con los tiempos de realización, un
diagrama de Gantt que recoge el tiempo y la dedicación de las tareas a lo largo del
diseño y desarrollo del sistema.
    \item\textbf{Cargas de trabajo: }
    Exposicion de las cargas de trabajo enfrentadas a la hora de elaborar el proyecto 
    \item\textbf{Presupuesto:}
    Descripción de los gastos que supondría elaborar este proyecto si se fuera a contratar a gente para desarrollarlo, así como los gastos en software.
    
    \item \textbf{ Conclusiones: }
    Conclusiones obtenidas al final el proyecto, así como posibles mejoras.
    \item\textbf{Valoración:}
Capítulo final donde, una vez finalizado el proyecto, se procede a tomar y exponer las
conclusiones del mismo. En esta parte también se han expuesto los resultados del producto final, así como una valoración personal y otra valoración ética.
\item\textbf{Repositorio Github}:
Enlace al respositorio de Github

\item\textbf{Anexo:}
Explicacion de las tecnologias usadas asi como un manual de uso
    
\end{itemize}