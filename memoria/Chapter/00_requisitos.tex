\chapter{Requisitos}\label{Int}
\fancyhf{}% Clear header/footer
\fancyhead[L]{PROYECTO DE FIN DE GRADO}
\fancyfoot[L]{\thepage} 
Es importante identificar los requisitos funcionales y no funcionales en un proyecto informático porque esto asegura que el sistema desarrollado cumpla con las expectativas y necesidades del cliente o usuario final.
\section{Funcionales}Requisitos funcionales:
Los requisitos funcionales describen qué debe hacer el sistema.
a continuación listaré los referentes a este proyecto.
\begin{enumerate}
    \item El bot de Telegram debe permitir que el usuario suba una foto y su ubicación
    \item La función de tratamiento de imágenes debe detectar el número de puntos en la imagen
    \item La aplicación web de Django debe mostrar la ubicación de la foto y el nivel de contaminación en un mapa
\end{enumerate}


\section{No funcionales}
Los requisitos no funcionales son aquellos que describen cómo debe ser el sistema en términos de características de calidad, como rendimiento, escalabilidad, usabilidad, entre otros.
\begin{enumerate}
    \item La aplicación debe ser fácil de usar e intuitiva
    \item La aplicación debe tener un tiempo de respuesta rápido y eficiente en términos de recursos
    \item La aplicación debe ser escalable para manejar una alta cantidad de usuarios y datos
\end{enumerate}

