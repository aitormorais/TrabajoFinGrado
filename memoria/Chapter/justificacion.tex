\chapter{JUSTIFICACIÓN}\label{Int}
\thispagestyle{fancy}
\fancyhead[LE]{\thechapter.JUSTIFICACION}
\section{JUSTIFICACIÓN}\label{justificación}
La justificación técnica de este proyecto se basa en la necesidad de una solución asequible y práctica para la detección de la contaminación atmosférica. Actualmente, la mayoría de la población considera que muchas de las soluciones existentes son caras y de difícil acceso, lo que limita su capacidad para controlar y comprender la calidad del aire en sus comunidades.

Además, la contaminación del aire es un problema global que afecta la salud humana y el medio ambiente, por lo que es importante que se desarrollen soluciones accesibles y asequibles para monitorear y comprender su impacto. Al permitir a los usuarios contribuir con datos sobre la calidad del aire, nuestro proyecto también promueve la participación ciudadana en la monitorización de la calidad del aire.

Desde el punto de vista técnico, este proyecto es viable gracias a la disponibilidad de tecnologías como las cámaras de los teléfonos móviles y las API de Google Maps, que permiten recopilar y procesar información sobre la calidad del aire de forma fácil y asequible. Además, el uso de un bot de Telegram y un sitio web construido en Django permite integrar y visualizar los resultados de forma sencilla, lo que facilita a los usuarios la comprensión y el uso de los datos.

En conclusión, la justificación técnica de este proyecto se basa en la necesidad de una solución asequible y práctica para la detección de la contaminación atmosférica y en la disponibilidad de tecnologías existentes que permiten su desarrollo y aplicación.