\chapter{Valoración}\label{val}
\thispagestyle{fancy}
\section{Ética}
Este proyecto tiene dos objetivos fácilmente identificables, siendo uno el poder acercar a la ciudadanía una herramienta fácil de utilizar para poder saber la calidad del aire en cualquier lado.
Poder tener acceso a la información relativa sobre la calidad del aire fácil y accesible.

Además, el uso de tecnología para la detección de la contaminación es una forma de promover la transparencia y la responsabilidad, ya que permite a los ciudadanos y a las autoridades tener acceso a información precisa y actualizada sobre los niveles de contaminación en tiempo real.

El principio de beneficencia está presente en dar una herramienta que es capaz de otorgarnos información sobre la calidad del aire. Esta herramienta puedo ayudar a la sociedad a concienciarse sobre la calidad del aire que respiran y así poder tomar medidas preventivas para proteger su salud y la de sus seres queridos.
También puede llevar a la ciudadanía a interesarse por las fuentes de contaminación en su entorno y así poder evitarlas, así como participar en la lucha contra la contaminación del aire.


\section{Personal}
Por último, cabe señalar que el progreso del proyecto ha tenido un impacto positivo en la vida intelectual y personal de los estudiantes. En primer lugar, los conocimientos adquiridos a lo largo de los años de formación se han aplicado a los múltiples aspectos relacionados con el proyecto. El desarrollo de un bot de Telegram y el tratamiento de las imágenes, han requerido, no obstante, el aprendizaje de estos temas.
Asimismo, ha sido enriquecedor diseñar y construir una solución que pueda servir de base o punto de partida para el eventual lanzamiento del sistema a gran escala en el futuro.