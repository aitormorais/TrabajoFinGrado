\chapter{Presupuesto}\label{int}
\thispagestyle{fancy}


\section{Desglose del presupuesto}
El coste de los recursos y gastos de personal necesarios para desarrollar la solución del sistema propuesto. Estos gastos se desglosan en dos partes: los costes de personal asociados a las tareas de dirección, diseño, programación; y el coste de los recursos necesarios o mínimos para su implantación.
\subsection{Gastos en personal}
Tomando una tasa media de 50 euros por hora, sabiendo que este proyecto tomo un tiempo dedicado al desarrollo software de alrededor de 350 horas, el precio sería de, 17500 euros, Suponiendo que haría falta un director de proyecto con una estimación de salario de 90 euros por hora sumariamos un total de 2250 euros por 25 horas de trabajo. Lo que nos deja un presupuesto de, 19750 euros para el personal.
\begin{table}[h]
\begin{center}
\begin{tabular}{| c | c | c | c | }
\hline
\multicolumn{4}{ |c| }{Gastos en personal} \\ \hline
Trabajador & Coste por horas €/h & Horas & Coste \\ \hline
Programador & 50 & 350 & 17500€ \\\hline
Director & 90 & 25 & 2250€ \\ \hline

\end{tabular}
\caption{Gasto en personal}
\label{tab:presupuesto personal}
\end{center}
\end{table}
\subsection{Gastos en software}
Las herramientas software empleadas para este proyecto tienen licencia open source, por lo tanto, no se precisa la realización de un pago para su uso.
El único pago que deberíamos de afrontar sería el despliegue de la web, el registro del dominio, así como el costo de certificado ssl.


\begin{itemize}
    \item Registro de dominio: entre 10 euros por año
    \item Hosting para la web: 150 euros por año
    \item Costo de certificado SSL: 70 euros por año
\end{itemize}

Teniendo en cuenta estos gastos nos deja un coste aproximado de 230 euros anuales
